%-------------------------------------------------------------------------------
%	SECTION TITLE
%-------------------------------------------------------------------------------
\cvsection{Bio}

% Ryan's research interests involve exploring circuit-level dynamics related to behavior \& memory using state-of-the-art computational techniques and methods. Ryan received his Ph.D. from the University of New Mexico, where he implemented machine-learning techniques such as independent component analysis to identify assemblies of neurons and recurrent neural networks to decode behavioral features from neural activity. In his current postdoctoral work at Cornell University, he uses machine learning methods to determine the level of coupling between multiple brain regions, which lead to a recent publication in the Journal Neuron. His future work is aimed at developing deep neural network forecasting models in order to create and improve the online detection of hippocampal events such as sharp-wave ripples and seizure-related interictal discharges. 



% Ryan received a bachelor's degree from Purdue University where he had a variety of research experiences, including exploring human navigation and memory in virtual environments as well as recording single unit hippocampal neurons in freely moving rodents. 

% During his PhD at the University of New Mexico, he investigated how neurodevelopmental disorders affect hippocampal place cell and sharp wave ripple activity. He is currently investigating the diversity of ripple-associated cell sequences using closed-loop optogenetics, silicon probe recordings, and advanced analytical tools.


% Ryan's research interests involve exploring circuit-level dynamics related to behavior \& memory using state-of-the-art computational techniques and methods. Ryan received a bachelor's degree from Purdue University where he had a variety of research experiences, including exploring human navigation and memory in virtual environments as well as recording single unit hippocampal neurons in freely moving rodents. During his PhD at the University of New Mexico, he investigated how neurodevelopmental disorders affect hippocampal place cell and sharp wave ripple activity. He is currently investigating the diversity of ripple-associated cell sequences using closed-loop optogenetics, silicon probe recordings, and advanced analytical tools. 

% Ryan Harvey, a skilled researcher with a strong scientific background, holds a Ph.D. from the University of New Mexico and has gained valuable experience through postdoctoral work at Cornell University. His expertise lies at the intersection of advanced technologies and cutting-edge methodologies. With a focus on machine learning and deep learning, Ryan actively explores the intricacies of neural networks and their applications. He has made notable contributions to the field, publishing research in esteemed journals such as Neuron and Current Biology. Recognized for his accomplishments, Ryan's proficiency in implementing advanced techniques in machine learning enables him to develop innovative solutions. His commitment to pushing the boundaries and his ability to harness transformative applications of these technologies make him an invaluable asset to any research team or organization seeking to leverage the power of machine learning.
    
\begin{cventries}
\cventry
    {} % Job title
    {} % Organization
    {} % Location
    {} % Date(s)
    {Ryan's research interests involve exploring circuit-level dynamics related to behavior \& memory using state-of-the-art computational techniques and methods. Ryan received his Ph.D. from the University of New Mexico, where he implemented machine-learning techniques such as independent component analysis to identify assemblies of neurons and recurrent neural networks to decode behavioral features from neural activity. In his current postdoctoral work at Cornell University, he uses machine learning methods to determine the level of coupling between multiple brain regions, which lead to a recent publication in the Journal Neuron. His future work is aimed at developing deep neural network forecasting models in order to create and improve the online detection of hippocampal events such as sharp-wave ripples and seizure-related interictal discharges.
    }
%---------------------------------------------------------
\end{cventries} 